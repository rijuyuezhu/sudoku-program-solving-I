\documentclass{article}
\usepackage{ctex}
\usepackage{hyperref}
\usepackage{graphicx}
\usepackage{float}
\usepackage{color, xcolor}
\usepackage{listings}
\title{Sudoku 数独游戏}
\author{黄文睿}
\date{2022 年 11 月}

\definecolor{mygreen}{rgb}{0,0.6,0}
\definecolor{mygray}{rgb}{0.5,0.5,0.5}
\definecolor{mymauve}{rgb}{0.58,0,0.82}
\lstset{ %
backgroundcolor=\color{white},   % choose the background color
basicstyle=\footnotesize\ttfamily,        % size of fonts used for the code
columns=fullflexible,
breaklines=true,                 % automatic line breaking only at whitespace
captionpos=b,                    % sets the caption-position to bottom
tabsize=4,
commentstyle=\color{mygreen},    % comment style
escapeinside={\%*}{*)},          % if you want to add LaTeX within your code
keywordstyle=\color{blue},       % keyword style
stringstyle=\color{mymauve}\ttfamily,     % string literal style
frame=single,
rulesepcolor=\color{red!20!green!20!blue!20},
% identifierstyle=\color{red},
language=c++,
}

\newcommand{\subsubsubsection}[1]{\paragraph{#1}\mbox{}\\}
\setcounter{secnumdepth}{4} % how many sectioning levels to assign numbers to
\setcounter{tocdepth}{4} % how many sectioning levels to show in ToC

\begin{document}

    \maketitle

    \tableofcontents

    
    \newpage

    \section{文件结构}
    \subsection{sudoku.h}

    \begin{lstlisting}
constexpr int BOARD_SIZE = 9;
constexpr int BLOCK_SIZE = 3;
    \end{lstlisting}

    表示数独总棋盘边长、数独九宫格边长。

    \begin{lstlisting}
constexpr int SCREEN_WIDTH = <>;
constexpr int SCREEN_HEIGHT = <>;
    \end{lstlisting}

    表示屏幕输出区的大小。(未最终确认大小)

    \begin{lstlisting}
constexpr int NUMBER_SIZE = 9;
    \end{lstlisting}

    表示数独中可填数的上界。

    \begin{lstlisting}
constexpr int KEY_TYPE_SIZE = 21;
    \end{lstlisting}

    表示按键类型的数量(见下方 Key\_type)。

    \begin{lstlisting}
enum Key_type {
    K_UP,
    K_DOWN,
    K_LEFT,
    K_RIGHT,
    K_REGAME,
    K_UNDO,
    K_REDO,
    K_ZERO,
    K_ONE,
    K_TWO,
    K_THREE,
    K_FOUR,
    K_FIVE,
    K_SIX,
    K_SEVEN,
    K_EIGHT,
    K_NINE,
    K_COMPLETE,
    K_COPY,
    K_QUIT,
    K_ENTER
};
    \end{lstlisting}

    定义不同类型的按键的名字。

    \begin{lstlisting}
struct Board {
    int init[BOARD_SIZE][BOARD_SIZE];
    int now[BOARD_SIZE][BOARD_SIZE];
};
    \end{lstlisting}

    用于储存数独信息的结构体。init 表示数独的初始情况,now 表示现在情况。游戏开始时 init=now。保留 init 信息方便判断操作的合法性。

    \begin{lstlisting}
typedef std::pair<int, int> Key_value;
    \end{lstlisting}

    用于保存 getch() 函数的返回值的类型。因为 getch() 函数对于一些按键需要获得两次,故使用一个 pair。

    \begin{lstlisting}
struct Setting_info {
    int difficulty;
    int open_hint;
    Key_value key_board[KEY_TYPE_SIZE];
};
    \end{lstlisting}

    用于保存当前配置信息的结构体。
    
    difficulty 表示难度,取值在 $1\sim 3$。

    open\_hint 表示是否开启提示,$1$ 表示开启,$0$ 表示不开启

    key\_board 存储的是所有按键类型对应的按键值。

    \begin{lstlisting}
typedef std::set<int> Hint;
    \end{lstlisting}

    定义 Hint 类型,表示提示信息。

    \begin{lstlisting}
struct Operation_node {
    int x, y, fr, to;
};
    \end{lstlisting}

    定义 Operation\_node 类型,表示操作序列中的一个节点,从而方便 Redo 和 Undo 操作。表示在 $(x, y)$ 将 $fr$ 改成 $to$。

    \begin{lstlisting}
typedef std::pair<int, int> Coord;
    \end{lstlisting}

    定义棋盘上的一个点。

    \begin{lstlisting}
typedef std::vector<Operation_node> Operation_sequence;
    \end{lstlisting}

    定义 Operation\_sequence 类型,表示操作序列,方便 Redo 和 Undo。

    \subsection{aid}
    
    \subsubsection{draw.cpp}

    \begin{lstlisting}
void Gotoxy(int x, int y);
    \end{lstlisting}
    
    将控制台光标放置到 $(x,y)$,从而方便 menu 函数定点输出。

    使用时应该注意,定位后的输出不能超过 SCREEN\_WIDTH 和\newline SCREEN\_HEIGHT。

    \begin{lstlisting}
void Clear_screen(void);
    \end{lstlisting}

    将屏幕全清。

    \begin{lstlisting}
void Set_color(int color);
    \end{lstlisting}

    修改之后的字体颜色。

    详情见\textcolor{blue}{\href{https://blog.csdn.net/wangxun20081008/article/details/115046817}{这篇文章}}。

    \begin{lstlisting}
void Set_window_color(int color);
    \end{lstlisting}

    设置整个窗口的字体前景色、背景色。

    详情见\textcolor{blue}{\href{https://blog.csdn.net/wangxun20081008/article/details/115046817}{这篇文章}}。

    \subsubsection{copypaste.cpp}

    \begin{lstlisting}
void Put_to_cp(std::string s);
    \end{lstlisting}

    将字符串 s 的值放置到剪切板里。

    \begin{lstlisting}
std::string Get_from_cp(void);
    \end{lstlisting}

    从剪切板里的内容获取并存放到 s 中。

    \subsubsection{stringboard.cpp}

    \begin{lstlisting}
std::string Board_to_string(Board NowBoard);
    \end{lstlisting}

    将 NowBoard 里的内容转换成字符串。

    该字符串如下:先行后列输出 init[][] 的内容后,再输出 now[][] 的内容。输出时先行后列,有空格与换行。

    如果 init 和 now 相同,则输出 81 个字符,仅包含 init[][] 的内容。

    \begin{lstlisting}
Board String_to_board(std::string s)
    \end{lstlisting}

    将 s 的内容转换成 Board 并返回。

    字符串规则同上。

    如果不符合规则,则返回的 Board 的 init[0][0] 为 -1。

    \subsubsection{get\_key.cpp}

    \begin{lstlisting}
Key_value Get_key(void);
    \end{lstlisting}

    获得一个按键。如果是单个键值的,则返回 $<value, 0>$,如果是两个键值的,则返回 $<value1, value2>$。注意需要把小写字母转化成大写字母。

    \begin{lstlisting}
std::string Get_keystring(Key_value key);
    \end{lstlisting}

    将 key 值转化用于描述的 std::string。这个 std::string 在文件内静态定义。如果未定义,返回空串。

    \subsection{generate}

    \subsubsection{generate\_lib.cpp}

    \begin{lstlisting}
bool Generate_lib(void);
    \end{lstlisting}

    从“棋盘文件”获得一个相应难度的棋盘,装载到 g\_game\_board 中。成功返回 1,否则返回 0。

    棋盘文件保存在 sudoku\_data/boards[123] 中,按照难度获取不同的文件。

    文件内保存格式为:81 个字符为一组,有空格和换行。详情请见 Put\_to\_cp;

    \subsubsection{generate\_saves.cpp}

    \begin{lstlisting}
bool Generate_saves(void);
    \end{lstlisting}

    获取存档位的信息,并装载到 g\_game\_board 中。成功返回 1,否则返回 0。

    存档文件保存在 sudoku\_data/saves 中。格式为 81 / 162 个为一组。(即为 Board\_to\_string)的格式。

    \subsubsection{generate\_cp.cpp}

    \begin{lstlisting}
bool Generate_cp(void);
    \end{lstlisting}

    从剪切板中获得棋盘,成功返回 1,否则返回 0。这里运用 Get\_from\_cp 与 String\_to\_board 函数完成。成功返回 1,失败返回 0。
    
    \subsubsection{generate\_menu.cpp}

    \begin{lstlisting}
void Generate_menu(void);
    \end{lstlisting}

    提供可视化菜单:需要有 "Start from game library","Start from last saves","Start from copy \& paste","Go back" 四个选项。使用 K\_UP 和 K\_DOWN 进行操作,使用 K\_ENTER 进行确认,同时可以使用 K\_ONE, K\_TWO,K\_THREE, K\_QUIT 等等。

    根据选项,分别调用 Generate\_lib,Generate\_saves,Generate\_cp,三个函数。

    \subsection{play}

    \subsubsection{game\_board.cpp}

    \begin{lstlisting}
Board g_game_board;
    \end{lstlisting}

    全局变量,表示目前的棋盘。

    \begin{lstlisting}
Operation_sequence g_operation_sequence;
    \end{lstlisting}

    全局变量,保存操作的序列。则 Undo 和 Redo 就可以借此完成。

    \begin{lstlisting}
int g_operation_sequence_id;
    \end{lstlisting}

    全局变量,和上面的 g\_operation\_sequence 一起使用,用来帮助 Redo。
    
    (即 Undo 时不立即弹出)

    \begin{lstlisting}
void Init_board(void);
    \end{lstlisting}

    初始化 g\_game\_board、g\_operation\_sequence、g\_operation\_sequence\_id。

    \subsubsection{game\_aid.cpp}

    \begin{lstlisting}
Hint Get_hint(int x, int y);
    \end{lstlisting}

    得到棋盘上 $(x, y)$ 这点的提示。注意 $x, y$ 的下标范围均为 $0\sim 8$。返回的 Hint 即 std::set<int> 为该点可填的数的集合。

    \begin{lstlisting}
bool Judge_legal(int x, int y);
    \end{lstlisting}

判断 $(x, y)$ 现在所填数是否合法。合法返回 1,不合法返回 0。

    \begin{lstlisting}
std::set<Coord> Get_illegal(void);
    \end{lstlisting}

    获得棋盘上不合法的棋子。

    \begin{lstlisting}
bool Judge_win();
    \end{lstlisting}

    判断游戏是否胜利,胜利返回 1,未胜利返回 0。

    胜利的条件:棋盘全部填满;符合数独的规则(这个可以调用 Judge\_legal 完成)。


    \subsubsection{operation}

    \subsubsubsection{move.cpp}

    \begin{lstlisting}
int g_cursor_x;
int g_cursor_y;
    \end{lstlisting}

    全局变量,为棋盘上的光标(取值均在 $0\sim 8$)。

    \begin{lstlisting}
void Move_init(void);
    \end{lstlisting}

    初始化光标。

    \begin{lstlisting}
bool Go_up(void);
bool Go_down(void);
bool Go_left(void);
bool Go_right(void);
bool Move_cursor(int x, int y);
    \end{lstlisting}

    移动光标。成功移动返回 1,否则返回 0。即改变 g\_cursor\_x 和 g\_cursor\_y 的值。如果到边界,则失败。除了 Move\_cursor 外,都会加入 operation\_sequence。

    \subsubsubsection{write\_number.cpp}

    \begin{lstlisting}
bool Write_number(int x, int y, int val, int ty);
    \end{lstlisting}

    将 $(x, y)$ 处的数更改为 val,成功返回 1,不成功返回 0。

    如果 $(x, y)$ 处为 init 中就存在的值(即不可填写的值),则失败。如果 ty 为 1,表示需要加入队列。

    \subsubsubsection{regame.cpp}

    \begin{lstlisting}
bool Regame(void);
    \end{lstlisting}

    发出重玩请求,需要确认,确实重玩返回 1,否则返回 0。即将 now 赋值为 init,再调用 Init\_board,再 Print\_board。

    \subsubsubsection{undo\_redo.cpp}

    \begin{lstlisting}
bool Undo(void);
bool Redo(void);
    \end{lstlisting}

    撤销、恢复操作。成功返回 1,失败返回 0。

    Undo 只需让 g\_operation\_sequence\_id 自减。若为 0,则失败。

    Redo 只需让 g\_operation\_sequence\_id 自增。若为 size-1,则失败。

    \subsubsubsection{solve.cpp}

    \begin{lstlisting}
Board Solve_game(Board NowBoard);
    \end{lstlisting}
    
    求解 NowBoard,返回解后 Board。这里计划采用 Dancing Link X 算法求解。如果无解,返回的 Board 的 now[0][0] 为-1。

    \subsubsubsection{complete\_game.cpp}

    \begin{lstlisting}
bool Complete_game(void);
    \end{lstlisting}

    一键完成游戏。成功返回 1,失败返回 0。无解则失败。

    \subsubsubsection{copy\_to\_paste.cpp}

    \begin{lstlisting}
bool Copy_to_paste(void);
    \end{lstlisting}

    将当前游戏复制到剪切板,成功返回 1,失败返回 0。一般不会失败。

    \subsubsubsection{save\_game.cpp}

    \begin{lstlisting}
bool Save_board(void);
    \end{lstlisting}

    将当前游戏保存到存档位中。成功返回 1,失败返回 0。存档不能正确打开则失败。

    存档文件保存在 sudoku\_data/saves 中。格式为 81 / 162 个为一组。(即为 Board\_to\_string)的格式。

    \subsubsection{print}

    \subsubsubsection{print\_board.cpp}

    \begin{lstlisting}
void Print_board(void);
    \end{lstlisting}

    初始化输出整个棋盘框架。(重刷)
    
    包括边界、初始数字、光标。

    \begin{lstlisting}
void Delete_cursor(int x, int y);
    \end{lstlisting}
    
    删除 $(x, y)$ 处的光标。(用空白覆盖)

    \begin{lstlisting}
void Add_cursor(int x, int y);
    \end{lstlisting}

    在 $(x, y)$ 处增添光标。

    \begin{lstlisting}
void Modify_number(int x, int y, int val);
    \end{lstlisting}

    将 $(x, y)$ 处的数字改为 $val$(覆盖)。
    
    \subsubsubsection{print\_hint.cpp}

    \begin{lstlisting}
void Print_hint(int x, int y);
    \end{lstlisting}

    在提示区打印 $(x, y)$ 的提示。

    \subsubsubsection{print\_message.cpp}

    \begin{lstlisting}
void Print_message(std::string s, int color);
    \end{lstlisting}
    
    在通知区打印通知。

    \subsubsection{run\_game.cpp}

    \begin{lstlisting}
void Run_game(void);
    \end{lstlisting}

    游戏的主体运行逻辑。
    
    首先调用 Init\_board,然后 Print\_board。接下来不断调用 Get\_key 等,然后根据操作分别调用 operation 中不同的函数。每次操作后,及时打印到屏幕。

    当对棋盘做出改变时,要及时写入文件(调用 Save\_Board 函数)

    \subsection{setting}

    \subsubsection{setting\_menu.cpp}

    \begin{lstlisting}
Setting_info g_setting_info;
    \end{lstlisting}

    保存配置信息的全局变量。

    \begin{lstlisting}
void Setting_menu(void);
    \end{lstlisting}

    显示设置菜单。需要有 "Difficulty", "Hint", "Keys", "Go back" 等几个选项。在每次调整后,自动写入文件(调用 Load\_setting)

    \subsubsection{difficulty.cpp}

    \begin{lstlisting}
void Modify_difficulty(int val);
    \end{lstlisting}

    将难度调整为 $val$。

    \subsubsection{open\_hint.cpp}

    \begin{lstlisting}
void Modify_open_hint(int val);
    \end{lstlisting}

    将提示开启与否调整为 $val$。

    \subsubsection{key\_board\_menu.cpp}

    \begin{lstlisting}
void Key_board_menu(void);
    \end{lstlisting}

    快捷键菜单。需要提供 KEY\_TYPE\_SIZE 种快捷键的更改操作。注意大小写字母都转化为大写字母。

    \begin{lstlisting}
bool Check_key(Key_type kt, Key_value key);
    \end{lstlisting}

    判断 kt 键位的值是否为 key。

    \subsubsection{key\_board.cpp}

    \begin{lstlisting}
void Modify_key_board(Key_type kt, Key_value key);
    \end{lstlisting}
    
    将 $kt$ 键位的值改为 $key$。
    
    \subsubsection{settingfile.cpp}

    \begin{lstlisting}
void Load_setting(void);
    \end{lstlisting}

    从 sudoku\_data/setting 文件中读取,并储存到 setting\_info 中。

    \begin{lstlisting}
void Save_setting(void);
    \end{lstlisting}

    将 setting\_info 的内容发送到 sudoku\_data/setting 中。
    \subsection{start.cpp}

    \begin{lstlisting}
void Start_menu(void);
    \end{lstlisting}

    开始菜单。提供 State Game(调用 Generate\_menu),Settings(调用 Setting\_menu),Exit 等功能。

    \subsection{main.cpp}

    \begin{lstlisting}
int main(void);
    \end{lstlisting}

    主函数。是整个程序的入口。
    
    首先先调用 Load\_setting() 读取配置,然后调用 Start\_menu(),进入开始页面。

\end{document}
